\documentclass[a4paper]{article}
\setlength{\parindent}{0pt}
\usepackage{amsmath,amssymb}
\usepackage{bm}
\usepackage{ctex}
\usepackage{enumerate}

\begin{document}

\title{反比例函数}

\maketitle

\section{反比例函数最初的样子}

如果你在读初中:

反比例函数最初的样子是 $y = \cfrac{1}{x}$,$x$ 的取值范围是 $x \neq 0$,$y$ 的取值范围是 $y \neq 0$。它的函数图像是两条曲线,人们常称之为双曲线。它的函数图像关于原点对称,是一个中心对称图形。它的函数图像关于直线 $y = x$ 和直线 $y = -x$ 对称,是轴对称图形,有两条对称轴。函数图像均位于第一、三象限,在第一象限内,函数图像随着 $x$ 的增大无限接近但达不到 $x$ 的正半轴,函数图像随着 $x$ 的减小无限接近但达不到 $y$ 的正半轴;在第三象限内,函数图像随着 $x$ 的减小无限接近但达不到 $x$ 的负半轴,函数图像随着 $x$ 的增大无限接近但达不到 $y$ 的负半轴;可见,函数 $y = \cfrac{1}{x}$ 的渐近线为直线 $x = 0$ 和 $y = 0$,两条渐近线交于原点。总之,它是一条悲伤的双曲线。

如果你在读高中:

反比例函数最初的样子是 $f(x) = \cfrac{1}{x}$,该函数的定义域为 $\{x | x \in \mathbf{R} \land x \neq 0\}$,值域为 $\{y | y \in \mathbf{R} \land y \neq 0\}$。它的函数图像是两条曲线,人们常称之为双曲线。它的函数图像关于原点对称,是一个中心对称图形,该函数是一个奇函数。它的函数图像关于直线 $y = x$ 和直线 $y = -x$ 对称,是轴对称图形,有两条对称轴。函数图像均位于第一、三象限,在第一象限内,函数图像随着 $x$ 的增大无限接近但达不到 $x$ 的正半轴,函数图像随着 $x$ 的减小无限接近但达不到 $y$ 的正半轴;在第三象限内,函数图像随着 $x$ 的减小无限接近但达不到 $x$ 的负半轴,函数图像随着 $x$ 的增大无限接近但达不到 $y$ 的负半轴;可见,函数 $y = \cfrac{1}{x}$ 的渐近线为直线 $x = 0$ 和 $y = 0$,两条渐近线交于原点。总之,它是一条悲伤的双曲线。

如果你在读大学:

反比例函数最初的样子是 $f(x) = \cfrac{1}{x}$,该函数的定义域为 $\{x | x \in \mathbf{R} \land x \neq 0\}$,值域为 $\{y | y \in \mathbf{R} \land y \neq 0\}$。它的函数图像是两条曲线,人们常称之为双曲线。它的函数图像关于原点对称,是一个中心对称图形,该函数是一个奇函数。它的函数图像关于直线 $y = x$ 和直线 $y = -x$ 对称,是轴对称图形,有两条对称轴。函数图像均位于第一、三象限,当 $x$(从 $0$)无限趋于 $-\infty$ 时,$y$ 从 $-\infty$ 无限趋于 $0$;当 $x$ 从 $-\infty$ 无限趋于 $0$ 时,$y$ 从 $0$ 无限趋于 $-\infty$;当 $x$ 从 $+\infty$ 无限趋于 $0$ 时,$y$ 从 $0$ 无限趋于 $+\infty$;当 $x$(从 $0$)无限趋于 $+\infty$ 时,$y$ 从 $+\infty$ 无限趋于 $0$;总之,我们有 $\lim\limits_{x \to -\infty} f(x) = 0$、$\lim\limits_{x \to 0^-} f(x) = -\infty$、$\lim\limits_{x \to 0+} f(x) = +\infty$ 和 $\lim\limits_{x \to +\infty} f(x) = 0$。可见,函数 $y = \cfrac{1}{x}$ 的渐近线为直线 $x = 0$ 和 $y = 0$,两条渐近线交于原点。

\section{修改 $x$ 导致函数图像左右平移}

牢记一句口诀:“左加右减”,即:将函数表达式中的 $x$ \textbf{全部}换成 $(x + h)$ 时($h > 0$),函数图像向左平移 $h$ 个单位;将 $x$ \textbf{全部}换成 $(x - h)$ 时($h > 0$),函数图像向右平移 $h$ 个单位。\textbf{请注意,该口诀对包括反比例函数、一次函数和二次函数在内的所有实数范围内的函数均成立。}

例如,将函数图像 $y = \cfrac{1}{x}$ 向左平移 1 个单位,我们可以得到 $y = \cfrac{1}{x + 1}$;将函数图像向左平移 $h$ 个单位($h > 0$),我们可以得到 $y = \cfrac{1}{x + h}$;将函数图像 $y = ax^2 + k$ 向左平移 $h$ 个单位($h > 0$),我们可以得到 $y = a(x - h)^2 + k$(当 $a \neq 0$ 时为二次函数的顶点式),请注意加括号。

如何理解“左加右减”?原本我的 $y = \cfrac{1}{x}$ 可以在 $x = 5$ 时得到 $y = \cfrac{1}{5}$ 的,但现在函数变成了 $y = \cfrac{1}{x + 1}$,我原本在 $x = 5$ 时就能得到的 $y = \cfrac{1}{5}$ 变成了在 $x = 4$ 时才能得到 $y = \cfrac{1}{5}$,因为把 $x = 4$ 代入 $y = \cfrac{1}{x + 1}$ 才能得到 $y = \cfrac{1}{5}$,也就是说,原来的 $\left(5, \cfrac{1}{5}\right)$ 变成了 $\left(4, \cfrac{1}{5}\right)$,那就说明函数图像往左平移了一位。所以,“左加右减”是对的。

\section{修改 $y$ 导致函数图像上下平移}

牢记一句口诀:“上加下减”,即:将形如 $y = f(x)$ 的函数表达式中等号的右侧加上 $k$ 时($k > 0$),函数图像向上平移 $k$ 个单位;将形如 $y = f(x)$ 的函数表达式中等号的右侧减去 $k$ 时($k > 0$),函数图像向下平移 $k$ 个单位。\textbf{请注意,该口诀对包括反比例函数、一次函数和二次函数在内的所有实数范围内的函数均成立;该口诀在使用时,请确保等号左侧只有一个 $y$,等号右侧不含 $y$。}

例如,将函数图像 $y = \cfrac{1}{x}$ 向上平移 1 个单位,我们可以得到 $y = \cfrac{1}{x} + 1$;将函数图像向上平移 $k$ 个单位($k > 0$),我们可以得到 $y = \cfrac{1}{x} + k$;将函数图像 $y = ax$(当 $a \neq 0$ 时为一次函数)向上平移 $b$ 个单位($b > 0$),我们可以得到 $y = ax + b$;将函数图像 $y = a(x - h)^2$向上平移 $k$ 个单位($k > 0$),我们可以得到 $y = a(x - h)^2 + k$(当 $a \neq 0$ 时为二次函数的顶点式)。

如何理解“上加下减”?把等式右边整体加上或减去一个正值,对应的 $y$ 值也就会相应地加上或减去一个正值,反映在图像上就是整体相应地向上或向下移动。

事实上,如果你有能力,你也可以类似“左加右减”那样定义修改 $y$ 导致函数图像上下平移,但可能就成为了“上减下加”,即:将函数表达式中的 $y$ \textbf{全部}换成 $(y - k)$ 时($k > 0$),函数图像向上平移 $k$ 个单位;将 $y$ \textbf{全部}换成 $(y + k)$ 时($k > 0$),函数图像向下平移 $k$ 个单位。不过,在日常数学中,人们一般将函数表达式写为 $y = f(x)$ 的形式,即等式左边只有一个 $y$,等式右边不含有 $y$,因此,更推荐的且人们更常用的是“上加下减”。

\section{复合平移}

理科中的复合,指的是多个独立的过程按一定的顺序结合在一起,例如由 $u(x) = \cfrac{1}{x}$ 和 $v(x) = x + 1$ 可以得到 $f(x) = u(v(x)) = \cfrac{1}{x + 1}$ 和 $g(x) = v(u(x)) = \cfrac{1}{x} + 1$,函数 $f(x)$ 和 $g(x)$ 被称为复合函数。

联合上述所学的知识,我们不难发现 $y = \cfrac{1}{x + 1} + 2$ 是 $y = \cfrac{1}{x}$ 先将 $x$(全部)换成 $x + 1$,再把 $\cfrac{1}{x + 1}$ 变成 $\cfrac{1}{x + 1} + 2$ 得来的,因此,函数 $y = \cfrac{1}{x + 1} + 2$ 的图像是 $y = \cfrac{1}{x}$ 先向左平移 $1$ 个单位(“左加”)再向上平移 $2$ 个单位(“上加”)得来的。

反之,如果将函数 $y = \cfrac{1}{x}$ 的图像先向右平移 $5$ 个单位,再向下平移 $4$ 个单位,就会得到函数 $y = \cfrac{1}{x - 5} - 4$。

一般地,函数 $y = \cfrac{1}{x - h} + k$ 的图像可以理解为由函数 $y = \cfrac{1}{x}$ 的图像先向右平移 $h$ 个单位,再向上平移 $k$ 个单位得来,其中向右平移 $-|h|$ 个单位意为向左平移 $|h|$ 个单位,向上平移 $-|k|$ 个单位意为向下平移 $|k|$ 个单位。由于是平移,我们同样可以得出以下四个结论:

\begin{enumerate}[(1)]
	\item 函数 $y = \cfrac{1}{x - h} + k$ 的图像关于点 $(h, k)$ 对称。
	\item 对直线 $y = x$ 先向右平移 $h$ 个单位,再向上平移 $k$ 个单位,可以得到 $y = (x - h) + k$,故函数 $y = \cfrac{1}{x - h} + k$ 的图像关于直线 $y = x - h + k$ 对称。
	\item 对直线 $y = -x$ 先向右平移 $h$ 个单位,再向上平移 $k$ 个单位,可以得到 $y = -(x - h) + k$,故函数 $y = \cfrac{1}{x - h} + k$ 的图像关于直线 $y = -x + h + k$ 对称。
	\item 函数 $y = \cfrac{1}{x - h} + k$ 的图像的渐近线为直线 $x = h$ 和直线 $y = k$,两条渐近线交于点 $(h, k)$。
\end{enumerate}

\section{伸缩}

令 $t \geqslant 1$,有如下函数图像变换:

\begin{enumerate}[(1)]
	\item 将函数表达式中的 $x$ \textbf{全部}换成 $tx$ 时,函数图像在 $x$ 方向上收缩为原来的 $\cfrac{1}{t}$。
	\item 将函数表达式中的 $x$ \textbf{全部}换成 $\cfrac{x}{t}$ 时,函数图像在 $x$ 方向上伸展为原来的 $t$ 倍。
	\item 将函数表达式中的 $x$ \textbf{全部}换成 $-\cfrac{x}{t}$ 时,函数图像在 $x$ 方向上伸展为原来的 $t$ 倍并关于 $y$ 轴对称。
	\item 将函数表达式中的 $x$ \textbf{全部}换成 $-tx$ 时,函数图像在 $x$ 方向上收缩为原来的 $\cfrac{1}{t}$ 并关于 $y$ 轴对称。
	\item 将形如 $y = f(x)$ 的函数表达式中等号的右侧变为 $tf(x)$ 时,函数图像在 $y$ 方向上伸展为原来的 $t$ 倍。
	\item 将形如 $y = f(x)$ 的函数表达式中等号的右侧变为 $\cfrac{f(x)}{t}$ 时,函数图像在 $y$ 方向上收缩为原来的 $\cfrac{1}{t}$。
	\item 将形如 $y = f(x)$ 的函数表达式中等号的右侧变为 $-\cfrac{f(x)}{t}$ 时,函数图像在 $y$ 方向上收缩为原来的 $\cfrac{1}{t}$ 并关于 $x$ 轴对称。
	\item 将形如 $y = f(x)$ 的函数表达式中等号的右侧变为 $-tf(x)$ 时,函数图像在 $y$ 方向上伸展为原来的 $t$ 倍并关于 $x$ 轴对称。
	\item 将函数表达式中的 $y$ \textbf{全部}换成 $ty$ 时,函数图像在 $y$ 方向上收缩为原来的 $\cfrac{1}{t}$。
	\item 将函数表达式中的 $y$ \textbf{全部}换成 $\cfrac{y}{t}$ 时,函数图像在 $y$ 方向上伸展为原来的 $t$ 倍。
	\item 将函数表达式中的 $y$ \textbf{全部}换成 $-\cfrac{y}{t}$ 时,函数图像在 $y$ 方向上伸展为原来的 $t$ 倍并关于 $x$ 轴对称。
	\item 将函数表达式中的 $y$ \textbf{全部}换成 $-ty$ 时,函数图像在 $y$ 方向上收缩为原来的 $\cfrac{1}{t}$ 并关于 $x$ 轴对称。
\end{enumerate}

以上思考过程,可类比“左加右减”。

\section{反比例函数的一般式}

一个反比例函数的一般式为 $y = \cfrac{ax + b}{cx + d}$($c \neq 0$),通常,我们会做向分母看齐的处理——$y = \cfrac{ax + b}{cx + d} = \cfrac{\cfrac{a}{c}(cx + d) - \cfrac{ad}{c} + b}{cx + d} = \cfrac{b - \cfrac{ad}{c}}{cx + d} + \cfrac{a}{c} = \cfrac{\cfrac{b}{c} - \cfrac{ad}{c^2}}{x + \cfrac{d}{c}} + \cfrac{a}{c} = \left(\cfrac{bc - ad}{c^2}\right)\cfrac{1}{x + \cfrac{d}{c}} + \cfrac{a}{c}$。因此,该一般式的函数图像可被认为是由函数 $y = \cfrac{1}{x}$ 先向左平移 $\cfrac{d}{c}$ 个单位(“左加”),再在 $y$ 方向上伸缩为向左平移后所得图像的 $\cfrac{bc - ad}{c^2}$,再将伸缩所得图像向上平移 $\cfrac{a}{c}$ 个单位得到的图像。类似地,我们同样可以得出以下两个结论:

\begin{enumerate}[(1)]
	\item 从原点 $(0, 0)$ 向左平移 $\cfrac{d}{c}$ 个单位得到 $\left(-\cfrac{d}{c}, 0\right)$(“左加右减”和“上加下减”仅适用于函数图像平移而不适用于点的平移),再将该点在 $y$ 方向上伸缩为该点的 $\cfrac{bc - ad}{c^2}$ 得到 $\left(-\cfrac{d}{c}, 0\right)$(没有变化),再将该点向上平移 $\cfrac{a}{c}$ 得到 $\left(-\cfrac{d}{c}, \cfrac{a}{c}\right)$。因此,函数 $y = \cfrac{ax + b}{cx + d}$($c \neq 0$)的图像关于点 $\left(-\cfrac{d}{c}, \cfrac{a}{c}\right)$ 对称。
	\item 函数 $y = \cfrac{ax + b}{cx + d}$($c \neq 0$)的图像的渐近线为直线 $x = -\cfrac{d}{c}$ 和直线 $y = \cfrac{a}{c}$,两条渐近线交于点 $\left(-\cfrac{d}{c}, \cfrac{a}{c}\right)$。
\end{enumerate}

由于函数 $y = \cfrac{ax + b}{cx + d}$($c \neq 0$)的图像可能涉及横坐标和纵坐标方向不等比例的伸缩,因此当且仅当 $\cfrac{bc - ad}{c^2} = \pm 1$ 时,该函数的图像保持有两条对称轴。

\begin{enumerate}[(1)]
	\item 对直线 $y = x$ 先向左平移 $\cfrac{d}{c}$ 个单位(“左加”),再在 $y$ 方向上伸缩为向左平移后所得图像的 $\cfrac{bc - ad}{c^2}$,再将伸缩所得图像向上平移 $\cfrac{a}{c}$ 个单位得到的图像。此时,我们得到 $y = \cfrac{bc - ad}{c^2}\left(x + \cfrac{d}{c}\right) + \cfrac{a}{c}$。由于 $\cfrac{bc - ad}{c^2} = \pm 1$,该函数简化为 $y = \pm\left(x + \cfrac{d}{c}\right) + \cfrac{a}{c}$。
	\item 对直线 $y = -x$ 先向左平移 $\cfrac{d}{c}$ 个单位(“左加”),再在 $y$ 方向上伸缩为向左平移后所得图像的 $\cfrac{bc - ad}{c^2}$,再将伸缩所得图像向上平移 $\cfrac{a}{c}$ 个单位得到的图像。此时,我们得到 $y = -\cfrac{bc - ad}{c^2}\left(x + \cfrac{d}{c}\right) + \cfrac{a}{c}$。由于 $\cfrac{bc - ad}{c^2} = \pm 1$,该函数简化为 $y = \mp\left(x + \cfrac{d}{c}\right) + \cfrac{a}{c}$。
\end{enumerate}

综上所述,当且仅当 $\cfrac{bc - ad}{c^2} = \pm 1$ 时,该函数的图像保持有两条对称轴,分别为 $y = x + \cfrac{a + d}{c}$ 和 $y = -x + \cfrac{a - d}{c}$。

\end{document}